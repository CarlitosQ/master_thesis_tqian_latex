%% LaTeX2e class for student theses
%% sections/conclusion.tex
%% 
%% Karlsruhe Institute of Technology
%% Institute for Program Structures and Data Organization
%% Chair for Software Design and Quality (SDQ)
%%
%% Dr.-Ing. Erik Burger
%% burger@kit.edu
%%
%% Version 1.3, 2016-12-29

\chapter{Conclusions \& Outlook}\label{ch:Conclusion}
Effect of feedback in polymer-based tunable DBR laser was investigated and compared with the case without feedback. The potential of linewidth reduction, bandwith enhancement and chirp reduction for such laser was demonstrated.

Theroy about feedback influenced laser cavity was modeled with the parameter $F=1+A+B$ which affects the laser linewidth and chirp, the modelling was adapted to the type of tunable laser used in this work. Numerical calculation showed siginificant difference compared to the laser without feedback, which gave a great potential for achieving better linewidth and bandwidth for polymer-based tunable laser. Measurements showed a linewidth reduction factor $F^2$ of $1.45$ was achieved which led to laser linewidth of $360 \ kHz$ with optical feedback, the corresponding chirp reduction by $F$ was also confirmed.

Feedback introduced spectra behavior such as self-modulation and undamped relaxation oscillation were observed in this work, the mechanism was studied by phase tuning measurements in combination with the detuned loading condition to explore the potential of extending the modulation bandwidth for polymer-based tunbale laser. Bandwith value of $10.6 \ GHz$ was achieved for laser without feedback, whileas maximum of $14.6 \ GHz$ was achieved when the laser is under feedback condition with self-modulation peaks appearing in the spectrum. The appearing of such self-modulation peaks in lasing spectra can be used as an indicator for setting the laser working in the extended bandwidth region.

Further improvement of linewidth and bandwidth for polymer-based tunable laser was explored by designing an on-chip controllable feedback laser combining the DBR laser structure with two types of feedback sections. First, short feedback section design successfully achieved the photon-photon resonance for the first time on poly-based tunable DBR laser. The appearing of the second peak in the frequency response permits the extended bandwidth value to $13.2 \ GHz$ in this work. Second, long feedback section with spiral structure possibily entered the weak feedback region led to phase dependent linewidth reduction and showed the lowest achieved linewidth value of $186 \ kHz$. However, siginificant linewidth reduction was not observed yet by this design. The problem so far seems like the feedback is not strong enough in such sprial design. Whether it is because of the high attenuation inside such waveguide structure or some fabrication issue or by coinscident choice of the characterized device need to be further studied.

The next steps building on these laser concepts are first the improvement of the achieved photon-photon resonance because so far the bandwidth value it achieved is still lower than the feedback introduced self-modulation case. Then further experiments cover the devices with other parameters in order to fully characterize this design.

% \newpage
% \section{Outlook}

