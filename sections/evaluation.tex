\chapter{Tunable Laser with on Chip Controllable Feedback}\label{ch:laser_with_on_chip_feedback}
\section{Design and Characterization}\label{sec:design_and_characterization}
The design of the on chip controllable feedback is targeting to the weak feedback region with phase dependent linewidth reduction and strong feedback region with coumpound cavity mode by ustilization of the Bragg grating, Multimode Interferometer (MMI), Variable Optical Attenuator (VOA) and Thin Film Filter (TFF) structures on our Polyboard platform. Design principle and the characterization of the chips are shown in the following sections.
% \begin{figure}[ht]
%     \centering
%     \includegraphics[width=10cm]{figures/feedback_region.PNG}
%     \caption{Feedback regimes for a laser diode.}
%     \label{feedback_region}
% \end{figure}


\subsection{Active and Passive Elements}
% As a variable optical attenuator (VOA) a 1x1 thermally tunable Multimode interferometer (MMI) is used (Fig.3.4.c). MMI is a multimode waveguide in which the light propagates in N modes. The various modes interfere with each other and produce an interference pattern in the MMI. This pattern allows at points of constructive interference to place output waveguide and thus to tap multiple outputs, each with a corresponding proportion of the total output power from an input signal. Depending on the design of the structure, the MMI can realize a different power division. On one side along the MMI is placed an electrode that allows using the thermo-optical effect. 
% We distinguish the states ON and OFF. In the OFF state, there is no heating of the electrodes and constructive interference takes place at the point where the output waveguide is. In ON-state, the interference pattern of the MMI is so affected, so that shifts through the partial change of the refractive index of the MMI, the position of constructive interference. Thus, a maximum interference point moves away from the position of the output waveguide and the output signal is tapped at reduced power (Fig.3.4.a and b). In this way a variable optical attenuation is created.
% An input waveguide with a width and height of 3.2 µm leads light into the interferometer. The height of the MMI (x-axis) also corresponds to 3.2 µm, the width is 18 µm and the length is 380 µm. The light that is leaving the MMI is received by an output waveguide. Height and width of the output waveguide coincide with the dimensions of the input waveguide. The transition of the input and output waveguide of the MMI is realized with taper sections that reduce the coupling losses. This VOA design was showing the best results compare to the literature [16-18].
The grating is designed to have its Bragg wavelength at $\lambda_B=1550 \ nm$.  Variable Optical Attenuator (VOA) is acheived with a $1\times 1$  thermally tunable MMI by placing an electrode on the side along the MMI which allows the thermo-optical effect. Thin Film Filter (TFF) acts as a high reflectivity mirror with its operating wavelength covers the grating Bragg wavelength, it can be simply inserted in the TFF slot thanks to our Polyboard technology. The characterization example of the VOA is shown in \autoref{fig:VOA_18321}. Maximum $-28.39 \ dB$ damping was achieved at current value of $54 \ mA$. 

% The Multimode Interferometer (MMI) is a multimode waveguide in which the light propagates in N modes. The various modes interfere with each other and produce an interference pattern in the MMI. This pattern allows at points of constructive interference to place output waveguide and thus to tap multiple outputs, each with a corresponding proportion of the total output power from an input signal. Depending on the design of the structure, the MMI can realize a different power division.

\begin{figure}[ht]
    \centering
    \includegraphics[width=0.55\linewidth]{figures/VOA_18321.png}
    \caption{Characterization of the VOA with maximum $-28.39 \ dB$ damping achieved at current value of $54 \ mA$.}
    \label{fig:VOA_18321}
\end{figure}

\subsection{Short Feedback Cavity} \label{subsec:short_feedback_cavity}
The on-chip short feedback cavity design is achieved by an additional waveguide and a TFF slot after the normal tunable laser, followed by a $1\times 2$ MMI to seperate the output port and the external cavity. In the external cavity, VOA and phase section are included. The chip design and the example of grating characterization is shown in \autoref{fig:grating_6559}. In order to exploit PPR, \autoref{eq:mode_spacing} is used to set the ideal cavity length. External cavity length of [3129.76, 3589,76, 3859.76, 4159.76, 4869.76, 5309.76, 6359.76] $\mu m$ are choosen according to the FSR plot shown in \autoref{fig:design_FSR}. The appearing of the ripples in the transmission and reflection curves indicates the existing of the strong reflection along with the grating. We calculated the spacing is corresponding to the external cavity and the distance between the MMI and the output port respectively, which indicates the reflection from the TFF slot is relatively high in this case.

\begin{figure}[ht]
    \centering
    \includegraphics[width=.55\linewidth]{figures/design_FSR.png}
    \caption{FSR versus different external cavity length, the red circle markers are corresponding to the FSR of [21.7, 19.8, 18.8, 17.8, 15.9, 14.9, 12.9] $GHz$.}
    \label{fig:design_FSR}
\end{figure}

\begin{figure}[ht]
    \centering
    \includegraphics[width=\linewidth]{figures/grating_6559.png}
    \caption{(a) Chip design of the laser with short external cavity controllable feedback, (b) grating characterization of the device, the spacing of the ripples in the transmission (black and blue) and reflection (red and green) curves are corresponding to the length of the external cavity and the distance between MMI and output port respectively.}
    \label{fig:grating_6559}
\end{figure}

\begin{figure}[ht]
    \centering
    \includegraphics[width=\linewidth]{figures/6559_short.png}
    \caption{Chip design of the laser with short external cavity controllable feedback.}
    \label{fig:grating_6559}
\end{figure}


% The parameter $F$ strongly depends on the phase of the externally reflected light $\phi_{ext}$ a very careful control of $\phi_{ext}$ is required for achieving a certain linewidth narrowing

Bandwidth enhancement by PPR is achieved with this short external cavity design. As shown in \autoref{fig:spectra_and_bandwidth_6559}, the appearing of the second resonance peak in the frequency response is clearly different from the ones we achieved in \autoref{fig:undamped_RO}. The mode spcing of $15.2 \ GHz$ is correspoing to the external cavity design of $4869.76 \ \mu m$ with FSR of $15.9 \ GHz$, the appearing of PPR with value of $14.8 \ GHz$ is slightly lower than the FSR value but in the same order. By tuning the phase section in the external cavity, the side mode shifts to the main mode and the PPR mode also moves toward the first peak which is CPR. Further tuning the phase section with the mode spacing close to the self mode locking range, the undamped RO starts to dominate and finally breaks the stable lasing condition.

The best achieved bandwidth value of $f_{3dB}=13.2 \ GHz$ with PPR is shown in \autoref{fig:spectra_and_bandwidth_6557}, with the external cavity length of $6359.76 \ \mu m$ and the designed FSR of $12.9 \ GHz$. The mode spacing between the main mode and the side mode is $11.5 \ GHz$ and the PPR is appearing at $11.6 \ GHz$ in the frequency response.

\begin{figure}[ht]
    \centering
    \includegraphics[width=.8\linewidth]{figures/spectra_and_bandwidth_6559.png}
    \caption{}
    \label{fig:spectra_and_bandwidth_6559}
\end{figure}

\begin{figure}[ht]
    \centering
    \includegraphics[width=\linewidth]{figures/spectrum_and_bandwidth_6557.png}
    \caption{The best achieved bandwidth value of $f_{3dB}=13.2 \ GHz$ with PPR. The external cavity length is $6359.76 \ \mu m$ corresponds to the FSR value of $12.9 \ GHz$. The mode spacing between the main mode and the side mode is $11.5 \ GHz$ and the PPR is appearing at $11.6 \ GHz$ in the frequency response.}
    \label{fig:spectra_and_bandwidth_6557}
\end{figure}

Chirp reduction is also observed by tuning the phase section in the external cavity, which is shown in \autoref{fig:chirp_6559} and \autoref{tab:chirp_6559}. Tuning the phase current from $10 \ mA$ to $14 \ mA$ we observed the increase of the Note that the current value in \autoref{tab:chirp_6559} is not exactly corresponding to the value in \autoref{fig:chirp_6559} because of the existance of the hysteresis.

\begin{figure}[ht]
    \centering
    \includegraphics[width=.7\linewidth]{figures/chirp_6559.png}
    \caption{}
    \label{fig:chirp_6559}
\end{figure}

\begin{table}[ht]
    \centering
    \caption{My caption}
    \label{tab:chirp_6559}
    \resizebox{\textwidth}{!}{%
    \begin{tabular}{@{}llllll@{}}
    \toprule
    \multirow{2}{*}{$I_{gain} \ [mA]$} & \multicolumn{5}{c}{$\alpha$}                                                                                \\ \cmidrule(l){2-6} 
                                      & $I_{phase}=10 \ mA$ & $I_{phase}=11 \ mA$ & $I_{phase}=12 \ mA$ & $I_{phase}=13 \ mA$ & $I_{phase}=14 \ mA$ \\ \midrule
    75                                & 1.725               & 2.063               & 2.22                & 2.624               & 3.247               \\
    75                                & 1.818               & 1.991               & 2.296               & 2.618               & 3.257               \\
    75                                & 1.818               & 2.07                & 2.246               & 2.614               & 3.253               \\
    75                                & 1.808               & 2.054               & 2.242               & 2.566               & 3.221               \\ \midrule
    Average                           & 1.79225             & 2.0445              & 2.251               & 2.6055              & 3.2445              \\ \bottomrule
    \end{tabular}%
    }
\end{table}

\subsection{Long Feedback Cavity} \label{subsec:long_feedback_cavity}
The on-chip long feedback cavity design is achieved by the spiral structure with the bending radius of $1500 \ \mu m$, it follows the same design principle as in \autoref{subsec:short_feedback_cavity}. The spiral structure has a variable design to achieve external cavity length of [39.28, 55.70, 81.33, 86.53, 100.75, 157.35] $\ mm$ by using \autoref{eq:F_weak_feedback} and \autoref{eq:F_strong_feedback}.

The chip design and the example of grating characterization is shown in \autoref{fig:grating_5742}. Similliar ripples were observed in the reflection curves but not in the transmission curves, it may because the attenuation inside the polymer waveguide after the spiral structure is relative high so that the reflected from the TFF slot got attenuated.
% \begin{figure}[!htb]
%     \centering
%     \includegraphics[width=0.6\linewidth]{figures/5742_spiral.png}
%     \caption{}
%     \label{fig:5742_spiral}
% \end{figure}

\begin{figure}[ht]
    \centering
    \includegraphics[width=\linewidth]{figures/grating_5742.png}
    \caption{(a) Chip design of the laser with long external cavity controllable feedback, (b) grating characterization of the device, the spacing of the ripples in the reflection (red and green) curves are corresponding to the distance between MMI and output port.}
    \label{fig:grating_5742}
\end{figure}

