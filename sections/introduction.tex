\chapter{Introduction}
\label{ch:Introduction}
The focus of this thesis is the study of the optical feedback effects on polymer-based tunable laser in order to improve its performance in linewidth and bandwidth. The tunable laser is constructed by a hybrid approach combining polymer-based photonic integrated circuits (PIC) and active Indium Phosphide (InP) components.

% \section{Tunable Lasers in Integrated Optics}
% The huge possibilities of digital interfaces have an impact on the optical line interfaces of the edge switches and the data-center gateways. Current 100G implementations based on dual-polarization quadrature phase-shift keying are becoming obsolete and we can expect the next migration from 100G to 400G and even further on to 1T based on higher quadrature amplitude modulation (QAM) formats. The aim of the research is to integrate a highly-functional coherent receiver chip, which includes an optical 90° hybrid as the frontend to convert the phase modulated signal into amplitude modulated signal detectable by photodiodes (PDs) [4-6], variable optical attenuators (VOAs) and a wavelength-tunable local oscillator (LO). Everything is done on PolyBoard (polymer-based hybrid platform). This technology allows inserting thin-film elements (TFEs) as polarizations rotators (PR) and polarization beam splitters (PBS). Lasing can be achieved by coupling an InP gain chip (GC) to a polymer Bragg grating. As 90° hybrids 2x4 MMI are used, as VOAs, thermally tunable 1x1 MMIs. This receiver component aims to serve as a potential candidate for 1 Tb/s transceivers for data-centers.

Coherent optical communication systems are gaining rapid importance because they allow to increase the speed (data rate) of transmission, by making use of the phase-predicatability of lasers with narrow spectral linewidth.

Semiconductor laser diodes with wide direct modulation bandwidth represent a relevant element to fulfill the continuously increasing need for low-cost optical communications systems with high bit-rate.

% Preliminary tests have shown our tunable DBR laser behaves differently under conditions of with and without feedback (see \autoref{ch:Characterization}. 

% \section{Hybrid Integration Using a Polymer-based Platform}
% The PolyBoard platform developed at the Heinrich Hertz Institute (HHI) is used to build hybrid devices allowing for a multitude of different functions like routing, switching, filtering and coupling [cite]. These passive structures are based on buried channel waveguides which have a refractive index contrast of $\Delta n=n_{core}-n_{clad}=0.03$. They support propagation of a single mode and are fabricated using the ZPU-12 series polymer material from ChemOptics consisting of polyfluorinated acrylic polymers [Cite]. They offer technical advantages compared to other systems, because of a simple, fast and low-temperature layer forming process, in contrast to epitaxy used for semiconductors and high-vacuum chemical vapor deposition (CVD) used in silicon photonics. The thermal properties of polymers make them ideal for the use in thermo-optic devices. They combine a high thermo-optic coefficient (TOC) with a low thermal conductivity, which results in highly efficient thermal tuning [cite].

For the design process a set of three masks is needed, corresponding to the electrode, waveguide and air trench geometry.

The goal of this work is to analyze and improve the characteristics of the laser under feedback condition. Parameters measured by different set-ups for the characterization are presented in \autoref{ch:normal_laser}. The samples produced based on this work are then characterized using optical transmission and reflection measurements. The measurements of these lasers, consisting of testing the designs, the application of different tuning principals and tuning experiments, are presented.